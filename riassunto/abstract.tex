\documentclass[a4paper]{article}

\usepackage[utf8]{inputenc}
\usepackage[italian]{babel}
\usepackage{graphicx}

\usepackage[backend=bibtex]{biblatex}

\addbibresource{../tesi/Biblio.bib}

\begin{document}

    \begin{center}
    {\huge{\bf Studio sull'incidentalità}}\\
    \vspace{3mm}
    {\huge{\bf stradale tramite dataset aperti}}\\
    \end{center}
    \vspace{3mm}
    \begin{center}
    {\bf Gabriele Padovani - Matr. 909165}\\
    \end{center}
    \vspace{2mm}
    \begin{center}
    {\bf Termine:}\\
    \end{center}


    % 1 Introduzione 5
    % 2 Origine dei dati 9
    % 2.1 Incidenti . . . . . . . . . . . . . . . . . . . . . . . . . . . . . . . . 9
    % 2.2 Autovelox . . . . . . . . . . . . . . . . . . . . . . . . . . . . . . . 10
    % 2.3 Patentati . . . . . . . . . . . . . . . . . . . . . . . . . . . . . . . 11
    % 2.4 Area C di Milano . . . . . . . . . . . . . . . . . . . . . . . . . . . 12
    % 2.5 Le zone di Milano . . . . . . . . . . . . . . . . . . . . . . . . . . 13
    % 2.6 Il meteo . . . . . . . . . . . . . . . . . . . . . . . . . . . . . . . . 13
    % 2.7 Trasporti pubblici . . . . . . . . . . . . . . . . . . . . . . . . . . 14
    % 2.8 Autostrade e manto stradale . . . . . . . . . . . . . . . . . . . . . 15
    % 2.9 Il turismo . . . . . . . . . . . . . . . . . . . . . . . . . . . . . . . 15
    % 2.10 Immagini del traffico . . . . . . . . . . . . . . . . . . . . . . . . . 16
    % 2.11 Dati mancanti . . . . . . . . . . . . . . . . . . . . . . . . . . . . . 16
    % 2.11.1 Il pav´e . . . . . . . . . . . . . . . . . . . . . . . . . . . . . 16
    % 2.11.2 Traffico stradale . . . . . . . . . . . . . . . . . . . . . . . 17
    % 2.11.3 Dati su strade, incroci, semafori . . . . . . . . . . . . . . 17

    % 3 Analisi sull’incidentalit`a 19

    % 3.1 Dati Geolocalizzati . . . . . . . . . . . . . . . . . . . . . . . . . . 19
    % 3.1.1 Distribuzione degli incidenti . . . . . . . . . . . . . . . . . 19
    % 3.1.2 Legame tra incidenti e trasporti pubblici . . . . . . . . . . 22
    % 3.1.3 Influenza del pav´e sull’incidentalit`a . . . . . . . . . . . . . 25
    % 3.1.4 Legame tra incidenti e autovelox . . . . . . . . . . . . . . 27

    % 3.2 Dati Istat su veicoli . . . . . . . . . . . . . . . . . . . . . . . . . . 32
    % 3.2.1 Incidenti per tipologia della strada . . . . . . . . . . . . . 32
    % 3.3 Dati Istat su conducenti e passeggeri . . . . . . . . . . . . . . . . 33
    % 3.3.1 Legami tra guida e sesso del conducente . . . . . . . . . . 33
    % 3.3.2 Numero di passeggeri . . . . . . . . . . . . . . . . . . . . 35
    % 3.3.3 Utilizzo del cellulare in incidenti . . . . . . . . . . . . . . 37
    % 3.4 Dati Istat su orari e mesi . . . . . . . . . . . . . . . . . . . . . . 38
    % 3.4.1 Differenze nei diversi orari della giornata . . . . . . . . . . 38
    % 3.4.2 Incidenti nelle fasce orarie della mattina . . . . . . . . . . 39
    % 3.4.3 Correlazione tra incidenti e traffico stradale . . . . . . . . 39
    % 3.4.4 Situazione in orari notturni . . . . . . . . . . . . . . . . . 41
    % 3.4.5 Vacanze nel mese di Agosto . . . . . . . . . . . . . . . . . 42
    % 3.4.6 Localit`a di mare nei mesi estivi . . . . . . . . . . . . . . . 46
    % 3.5 Dati Istat su tipi di incidenti e incroci . . . . . . . . . . . . . . . 49
    % 3.5.1 Frequenza di incidenti per tipologia . . . . . . . . . . . . 49
    % 3.5.2 Tipologie di strada e indice di mortalit`a . . . . . . . . . . 51
    % 3.5.3 Feriti per tipologia di sinistro . . . . . . . . . . . . . . . . 52
    % 3.5.4 Coinvolgimento di pedoni in incidenti . . . . . . . . . . . 54
    % 3.5.5 Et`a dei pedoni coinvolti . . . . . . . . . . . . . . . . . . . 56
    % 3.5.6 Et`a dei conducenti . . . . . . . . . . . . . . . . . . . . . . 59

    % 3.6 Dati ACI . . . . . . . . . . . . . . . . . . . . . . . . . . . . . . . 61
    % 3.6.1 Quadro delle regioni italiane . . . . . . . . . . . . . . . . 61
    % 3.6.2 Differenze di incidenti tra mesi estivi e invernali . . . . . 64
    % 3.6.3 Quadro delle province lombarde e laziali . . . . . . . . . . 66
    % 3.6.4 Approfondimento sulla regione Lombardia . . . . . . . . . 66
    % 3.6.5 Correlazione tra incidenti e numero di feriti . . . . . . . . 67
    % 3.6.6 Quadro delle strade pi`u pericolose . . . . . . . . . . . . . 70
    % 3.6.7 Approfondimento delle tratte pi`u pericolose . . . . . . . . 71
    % 3.6.8 Evoluzione cronologica degli incidenti . . . . . . . . . . . 72

    % 3.7 Meteo a Milano . . . . . . . . . . . . . . . . . . . . . . . . . . . . 78
    % 3.7.1 Correlazione tra fattori atmosferici e incidentalit`a . . . . . 78

\section{Introduzione}

Nel documento di tesi si è esplorato che tipo di analisi è possibile realizzare, 
avendo a disposizione sufficiente quantità di dati liberi. 
Gli argomenti del lavoro, si concentrano in particolare sull'ambito degli incidenti stradali, 
con enfasi sulle posizioni di questi ultimi, sia dal punto di vista delle coordinate 
geografiche, sia da quello delle strade con maggiore numero di sinistri. 

\section{Origine dei dati}

La maggior parte delle informazioni utilizzate, generalmente raggruppate in dataset, sono 
state reperite nei principali siti di open data, come quello del comune di 
Milano\footnote{\url{https://dati.comune.milano.it/}}, o 
quello del ministero dei trasporti\footnote{\url{https://www.mit.gov.it/}}.

D'altra parte, i file principali, contenenti i dati riguardanti gli incidenti, sono 
stati reperiti sul sito 
dell'Istat\footnote{\url{https://www.istat.it/it/archivio/87539}}, e su quello 
dell'Aci \cite{ACI:1}.

Le informazioni riguardanti le coordinate degli incidenti avvenuti a Milano, provengono invece 
dal giornale online TheSubmarine \cite{SUBMARINE:1}, 
che ha ottenuto il rilascio di una parte di questi dati, 
normalmente oscurati per rispettare la privacy degli individui coinvolti nei sinistri.

\section{Analisi dei dati geolocalizzati}

Per quanto le analisi riguardanti gli incidenti che dispongono di localizzazione, 
si sono incentrate soprattutto sulla relazione tra incidentalità e pavimentazione 
della strada, o sulla rapporto tra numero di incidenti e presenza di linee di 
trasporti pubblici, o ancora sulla differenza di sinistri nelle vicinanze di autovelox. 

\section{Analisi con dati Istat}

Per quanto riguarda le analisi realizzate sui dataset provenienti dal sito Istat, 
ci si è concentrati sull'evoluzione dell'incidentalità nei diversi orari della giornata, 
e allo stesso modo al passare dei mesi. 

Si è inoltre controllato se esistessero tipologie di strade e incroci che favorissero 
un numero maggiore di sinistri, e allo stesso modo, se alcuni di questi presentassero 
un valore anomalo di pedoni coinvolti.

Infine, si è tentato di individuare l'esistenza di fattori di distrazione del 
conducente, in particolare nella forma dei telefoni cellulari e nella presenza 
di altri passeggeri a bordo. 

\section{Analisi con dati Aci}

I dataset provenienti dal sito Aci, d'altra parte, sono stati oggetto di varie analisi per 
quanto riguarda il luogo dell'incidente, inteso come il nome della strada o la provincia 
in cui questo è avvenuto. 

Più nello specifico, si sono create varie rappresentazioni, incentrate sugli incidenti 
per regione, per le province di Lombardia e Lazio, e per le principali autostrade, sia in 
prossimità di Milano sia in generale in Italia.

Infine, riprendendo alcune delle analisi realizzate su dati Istat, si sono calcolate alcune 
informazioni, incentrate soprattutto sull'evoluzione degli incidenti al cambiare dell'orario 
o del mese dell'anno.

\section{Analisi con dati Meteo}

I dataset riguardanti le informazioni meteo, sono stati utilizzati per un'analisi sulla 
correlazione tra i fattori atmosferici e l'incidentalità in 
prossimità della città di Milano. 
In particolare, ci si è concentrati sulle informazioni riguardanti la temperatura, 
l'umidità e la velocità del vento. 
Causa la bassa precisione dei dati utilizzati, tuttavia, questo capitolo è 
da intendersi come una dimostrazione di quanto, un calcolo realizzato tra due dataset 
scollegati, possa comunque fruttare un risultato plausibile ma infondato. 

D'altra parte, non è da escludere che, tramite analisi più approfondite, 
sia possibile ottenere risultati validi.

\section{Conclusioni}

\printbibliography

\end{document}